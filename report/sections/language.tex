\section{Language \texttt{fun}}
\subsection{Introduction}
Let us define the sample language we will be using throughout the paper to later describe compilation and semantics.
\subsection{Syntax}
The BNF of our language is as follows:

\begin{grammar}

<expr> ::= <ident>
\alt <int>
\alt `\lambda' <ident> `->' <expr>
\alt <expr> <expr>
\alt `let' <ident> `=' <expr> `in' <expr>
\alt <expr> <bop> <expr>
\alt `if' <expr> `is 0 then' <expr> `else' <expr>
\alt `fix' <expr>

<bop> ::= `+'
\alt `*'
\alt `-'

<int> ::= (integer literals, e.g., 0, 1, 2, ...)  

<ident> ::= (identifiers, e.g., x, y, foo, ...)  

\end{grammar}

\subsection{Haskell datatype}
For the implementation, instead of using identifiers, we use De Bruijn indexing to refer to variables.
The datatype of our language is defined with a GADT that captures the return type
of each constructor definition:

\begin{lstlisting}
data Expr a where
  Var :: Int -> Expr a
  Int :: Int -> Expr Int
  Abs :: Expr b -> Expr (a -> b)
  App :: Expr (a -> b) -> Expr a -> Expr b
  Let :: Expr a -> Expr b -> Expr b
  Bop :: BopE -> Expr Int -> Expr Int -> Expr Int
  IfZ :: Expr Int -> Expr a -> Expr a -> Expr a
  FixP :: Expr (a -> a) -> Expr a

data BopE
  = Add
  | Mul
  | Sub
\end{lstlisting}

% fun+ =
% \begin{grammar}
%
% <expr> ::= <ident>
% \alt <int>
% \alt `\\' <ident> `->' <expr>
% \alt <expr> <expr>
% \alt `let' <ident> `=' <expr> `in' <expr>
% \alt <expr> <bop> <expr>
% \alt <op> <expr>
% \alt `if' <expr> `is 0 then' <expr> `else' <expr>
% \alt `fix' <expr>
% \alt `('<expr>`,' <expr>`)'
%
% <bop> ::= `+'
% \alt `*'
% \alt `-'
%
% <op> ::= `fst'
% \alt `snd'
%
% <int> ::= (integer literals, e.g., 0, 1, 2, ...)  
%
% <ident> ::= (identifiers, e.g., x, y, foo, ...)  
%
% \end{grammar}
%
% \begin{lstlisting}
% data Expr a where
%   Var :: Int -> Expr a
%   Int :: Int -> Expr Int
%   Abs :: Expr b -> Expr (a -> b)
%   App :: Expr (a -> b) -> Expr a -> Expr b
%   Let :: Expr a -> Expr b -> Expr b
%   Bop :: BopE -> Expr Int -> Expr Int -> Expr Int
%   Op :: OpE -> Expr a -> Expr a
%   IfZ :: Expr Int -> Expr a -> Expr a -> Expr a
%   FixP :: Expr (a -> a) -> Expr a
%   Tup :: Expr a -> Expr b -> Expr (a, b)
%
% data BopE
%   = Add
%   | Mul
%   | Sub
%
% data OpE
%   = Fst
%   | Snd
% \end{lstlisting}
