\section{Language \texttt{fun}}
\subsection{Introduction}
Let us define the sample language we will be using throughout the paper to later describe compilation and semantics.
\subsection{Syntax}
The BNF of our language is as follows:

\begin{grammar}

<expr> ::= <ident>
\alt <int>
\alt `\lambda' <ident> `\to' <expr>
\alt <expr> <expr>
\alt `let' <ident> `=' <expr> `in' <expr>
\alt <expr> <bop> <expr>
\alt `if' <expr> `is 0 then' <expr> `else' <expr>
\alt `fix' <expr>

<bop> ::= `+'
\alt `*'
\alt `-'

<int> ::= (integer literals, e.g., 0, 1, 2, ...)

<ident> ::= (identifiers, e.g., x, y, foo, ...)

\end{grammar}

\subsection{Haskell datatype}
For the implementation, instead of using identifiers, we use De Bruijn indexing to refer to variables.
The datatype of our language is defined with a GADT that captures the return type
of each constructor definition:

\begin{lstlisting}[mathescape]
data Expr a where
  Var :: Int $\to$ Expr a
  Int :: Int $\to$ Expr Int
  Abs :: Expr b $\to$ Expr (a $\to$ b)
  App :: Expr (a $\to$ b) $\to$ Expr a $\to$ Expr b
  Let :: Expr a $\to$ Expr b $\to$ Expr b
  Bop :: BopE $\to$ Expr Int $\to$ Expr Int $\to$ Expr Int
  IfZ :: Expr Int $\to$ Expr a $\to$ Expr a $\to$ Expr a
  FixP :: Expr (a $\to$ a) $\to$ Expr a

data BopE
  = Add
  | Mul
  | Sub
\end{lstlisting}
