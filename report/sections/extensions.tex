\section{Extensions}
\subsection{Tuples}
Tuples are an easy addition to our language.

We can extend our GADT to support tuples easily, and lets assume \texttt{fun} has been extended to parse 
tuples as well.

\begin{lstlisting}
data Expr a where
  ...
  Tup :: Expr a -> Expr b -> Expr (a, b)
\end{lstlisting}

Tuples are akin to values. They can be passed as arguments and returned from functions, so we will need
to keep them in our stack. For that reason, we shall extend \texttt{Val} with a constructor for them.

\begin{lstlisting}
data Val
  ...
  | VTuple (Val, Val)
\end{lstlisting}

We also need something that tells the abstract machine that it should build this value, that it should
glue these two arbitrary things together. We need a new instruction...

\begin{lstlisting}
data Instr
  ...
  | TUP
  ...
\end{lstlisting}

And a new compilation rule.

\[ \mathcal{C}((M, N)) = \mathcal{C}(M);\; \mathcal{C}(N);\; \texttt{TUP} \]

Now that we have a way to construct tuples, we also need a way to destruct them. We can extend our language
with a "unary operation" construct, and encode \texttt{fst} and \texttt{snd} with it. We also need the corresponding
instructions such that the interpreter can run them.

\begin{lstlisting}
data Expr a where
  ...
  Op :: OpE -> Expr a -> Expr a
  ...

data OpE
  = Fst
  | Snd

data Instr
  ...
  | FST
  | SND
  ...
\end{lstlisting}

We are pretty much done! We just need to define how the interpreter will handle the new cases, which are pretty straight-forward:

\begin{table}[h]
    \centering
    \small
    \begin{tabular}{|l|l|l||l|l|l|}
        \hline
        \multicolumn{3}{|c||}{\textbf{Before}} & \multicolumn{3}{c|}{\textbf{After}} \\
        \hline
        \textbf{Code} & \textbf{Env} & \textbf{Stack} & \textbf{Code} & \textbf{Env} & \textbf{Stack} \\
        \hline
        TUP : \(c\) & \( e \) & \( v_2 \) : \( v_1 \) : \( s \) & \( c \) & \( e \) & VTuple(\( v_1, v_2 \)) : \( s \) \\
        FST : \(c\) & \( e \) & VTuple(\( v_1, v_2 \)) : \( s \) & \( c \) & \( v \) : \( e \) & \( v_1 \) : \( s \) \\
        SND : \(c\) & \( e \) & VTuple(\( v_1, v_2 \)) : \( s \) & \( c \) & \( v \) : \( e \) & \( v_2 \) : \( s \) \\
        \hline
    \end{tabular}
    \caption{Tuple Machine Transitions}
\end{table}

\subsection{Untyped Variants}

Variants are a type of data structure that can represent one of several possible values,
each tagged with a unique constructor. We should be able to match on each constructor individually - this is the destructor of variants.
They are also akin to values, just like tuples as before.

A way to encode variants in our language would be to extend the AST with the following:

\begin{lstlisting}
data Expr a where
  ...
  Variant :: (String, Expr a) -> Expr a -- (name, expression) creates a variant.
  Match :: (Expr a, [(String, Expr a)]) -> Expr a -- match the input with a list of constructors;
                                                  -- "Var 0" will contain the inner expression.
\end{lstlisting}

As before, we need some way to tell the abstract machine to \textbf{construct} the variant, and another to start the \textbf{destruction} process.
Let's extend our instruction set and value data type with the required elements:

\begin{lstlisting}
data Val
  ...
  | VVariant String Val

data Instr
  ...
  | VARIANT String
  | MATCH [(String, [Instr])]
  ...
\end{lstlisting}

Compilation is a bit harder for the match case, 
we have to use some list magic and compile the inner expressions of the branches.

\[ \mathcal{C}(\texttt{Variant (Name, E)}) = \mathcal{C}(E);\; \texttt{VARIANT Name} \]
\[ \mathcal{C}(\texttt{match E with branches}) = \mathcal{C}(E);\; \texttt{MATCH branches'} \]
\[  \texttt{where branches'} = map\; (\lambda (\texttt{name},\; \texttt{branch}) \to (\texttt{name},\; \mathcal{C}(\texttt{branch})))\; \texttt{branches} \]

Now, the fun part! Interpreting is quite easy, the interesting case is for the \texttt{match} expression:

\begin{table}[h]
    \centering
    \footnotesize
    \begin{tabular}{|l|l|l||l|l|l|}
        \hline
        \multicolumn{3}{|c||}{\textbf{Before}} & \multicolumn{3}{c|}{\textbf{After}} \\
        \hline
        \textbf{Code} & \textbf{Env} & \textbf{Stack} & \textbf{Code} & \textbf{Env} & \textbf{Stack} \\
        \hline
        VARIANT name : \(c\) & \( e \) & \( v \) : \( s \) & \( c \) & \( e \) & VVariant(\( name, v \)) : \( s \) \\
        MATCH branches : \(c\) & \( e \) & VVariant \( name, v \) : \( s \) & branches \( !! \) name : \( c \) & \( v \) : \( e \) & \( s \) \\
        \hline
    \end{tabular}
    \caption{Variant Machine Transitions}
\end{table}

The operator \( !! \) is the lookup operation for lists. If we find the name of the constructor 
we are matching for in the association list, then we execute the code relevant to that branch.
If we don't find a match, then we should exit with an error, warning the user that the pattern-match they wrote was 
not extensive.

We can now encode lists, peano numbers, and a whole lot more in our language. The combination of tuples and variants
is extremely expressive!

As this is quite a bit abstract, I urge you to check the tests in CITE APPENDIX for a better intuition.

\subsubsection{Why both Tuples and Variants}

While it is true that we can encode tuples using variants, I personally found that having
an abstraction specifically for tuples in our language is a lot nicer than having to write something like this
to encode a pair:

\begin{lstlisting}
let pair = Variant "Pair" (Variant "First" (Int 3), Variant "Second" (Int 5))
\end{lstlisting}

As it stands, to write functions that operate on this pair, you would have to write two match expressions.
This is just a bit cumbersome, and for the purposes of this paper, it was also a good opportunity to show how easy
tuples are to implement.

\subsubsection{Encoding Records with Variants}

As a side note, variants can be used to implement records, they can be seen as the opposite of one another: 
\begin{itemize}
  \item a variant is this field \textit{or} that field \textit{or} that field;
  \item a record is this field \textit{and} that field \textit{and} that field.
\end{itemize}

So, if we "invert" a variant, we should get something that behaves like a record.

We can encode records using variants like so (CITE Principles of Programming Languages):
\[
\{l_1 = e_1; \dots; l_n = e_n\} = 
(\lambda k_1 \to \dots \lambda k_n \to \lambda s \to 
\text{match } s \text{ with } 
\begin{cases}
    'l_1(x) \to k_1 \\
    \quad \vdots \quad \\
    'l_n(x) \to k_n
\end{cases}
)(e_1) \dots (e_n)
\]
\[
e.l_k = e \;\; 'l_k(0)
\]

A bit complicated, but it works! See (CITE Principles of Programming Languages) for further explanation.
