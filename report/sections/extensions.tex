\section{Extensions}
\subsection{Tuples}
Tuples are an easy addition to our language.

We can extend our GADT to support tuples easily, and lets assume \texttt{fun} has been extended to parse 
tuples as well.

\begin{lstlisting}
data Expr a where
  ...
  Tup :: Expr a -> Expr b -> Expr (a, b)
\end{lstlisting}

Tuples are akin to values. They can be passed as arguments and returned from functions, se we will need
to keep them in our stack. For that reason, we shall extend \texttt{Val} with a contructor for them.

\begin{lstlisting}
data Val
  ...
  | VTuple (Val, Val)
\end{lstlisting}

We also need something that tells the abstract machine that it should build this value, that it should
glue these two arbitrary things together. We need a new instruction and a new compilation rule for that.

\begin{lstlisting}
data Instr
  ...
  | TUP
  ...
\end{lstlisting}


\[ \mathcal{C}((M, N)) & = \mathcal{C}(M);\; \mathcal{C}(N);\; \texttt{TUP} \]

Now that we have a way to contruct tuples, we also need a way to destruct them. We can extend our language
with a "unary operation" construct, and encode \texttt{fst} and \texttt{snd} with it. We also need the corresponding
instructions such that the interpreter can run them.

\begin{lstlisting}
data Expr a where
  ...
  Op :: OpE -> Expr a -> Expr a
  ...

data OpE
  = Fst
  | Snd

data Instr
  ...
  | FST
  | SND
  ...
\end{lstlisting}

We are pretty much done! We just need to define how the interpreter will handle the new cases, which are pretty straight-forward:

\begin{table}[h]
    \centering
    \small
    \begin{tabular}{|l|l|l||l|l|l|}
        \hline
        \multicolumn{3}{|c||}{\textbf{Before}} & \multicolumn{3}{c|}{\textbf{After}} \\
        \hline
        \textbf{Code} & \textbf{Env} & \textbf{Stack} & \textbf{Code} & \textbf{Env} & \textbf{Stack} \\
        \hline
        TUP : \(c\) & \( e \) & \( v_2 \) : \( v_1 \) : \( s \) & \( c \) & \( e \) & VTuple(\( v_1, v_2 \)) : \( s \) \\
        FST : \(c\) & \( e \) & VTuple(\( v_1, v_2 \)) : \( s \) & \( c \) & \( v \) : \( e \) & \( v_1 \) : \( s \) \\
        SND : \(c\) & \( e \) & VTuple(\( v_1, v_2 \)) : \( s \) & \( c \) & \( v \) : \( e \) & \( v_2 \) : \( s \) \\
        \hline
    \end{tabular}
    \caption{TAP Machine Transitions}
\end{table}

\subsection{Simple Type-Inference}
