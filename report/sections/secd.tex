\section{SECD Machine}
\subsection{Introduction}
The SECD machine was introduced in 1964 by Landin. It has a simple instruction set that operates
on stacks. You can think about this instruction set as a virtual machine of sorts, if you translate
your language into this instruction set, through a process called \textbf{compilation},
then there is a standard way of running said instructions. Hopefully, given your compiled program is
correct, the list of instructions is \textbf{interpreted}, producing the expected behaviour of the term written in your language.

The list of instructions is just the first part, to \textbf{interpret} them, the SECD machine keeps 4 data structures in memory, 
and sometimes another helper data structure to support function applications.

\subsection{Data Structures}
\subsubsection{S: Stack}
The Stack holds intermediate results from our interpreting process. It is a list of \textbf{values}.
It is also where our final output will be stored.

The stack is a list of the \texttt{Val} data type:
\begin{lstlisting}
data Val
  = VInt Int
  | Addr Int -- corresponds to the position of either a variable or a function inside the Env
\end{lstlisting}

\subsubsection{E: Environment}
The Environment associates the values that are bound to variables. It can be seen as a simple list of values, where the indexes are refer to the
variables themselves.
\subsubsection{C: Code}
This is the list of instructions that we compiled previously, and are currently running.
\subsubsection{D: Dump}
The dump stores the contents of the other 3 registers temporarily, while a function is executing, for example.
You can see these function calls as \textbf{detours} from the original execution. When they are finished, 
we can pop the head of the Dump to go retrieve the main execution path.

We will see later that we can combine the Stack and Dump into just one data structure.
\subsubsection{Closures on a Store}
Closures represent function applications that are yet to be processed. As environments are very dynamic and can change quickly, when creating a Closure, the current environment
is captured inside of it. This makes sure the function has access to the unbounded variables it needs.

Closures can appear on both the Stack and the Dump, but for this first implementation we will keep them in a separate place in memory,
a Store.

\subsection{Default Instructions}
The default set of instructions is relatively small:
\begin{lstlisting}
data Instr
  = LD Int -- loads the variable at the specified index 
  | LDC Int -- loads an integer constant
  | CLO [Instr] -- loads a function
  | FIX [Instr] -- loads a recursive function
  | AP -- applies a function to an argument
  | RTN -- returns from function application
  | IF [Instr] [Instr] -- tests a condition and selects one branch accordingly
  | JOIN -- returns from whichever branch IF selected
  | ADD -- adds two values
  | SUB -- subtracts two values
  | MUL -- multiplies two values
  | HALT -- halts execution
\end{lstlisting}

\subsection{Compilation from \texttt{fun}}
We define the compilation function $ \mathcal{C} $, with type $ \texttt{Expr a} -> \texttt{[Instr]} $:
\[
\begin{aligned}
    \mathcal{C}(n) &= \mathtt{LDC}\; n \\
    \mathcal{C}(\underline{n}) &= \mathtt{LD}\; n \\
    \mathcal{C}(\mathtt{\lambda\; \_ \to M}) &= \mathtt{CLO}\; (\mathcal{C}(\mathtt{M}); \mathtt{RTN}) \\
    \mathcal{C}(\mathtt{M\; N}) &= \mathcal{C}(\mathtt{M});\; \mathcal{C}(\mathtt{N});\; \mathtt{AP} \\
    \mathcal{C}(\mathtt{let\; \_ = M\; in\; N}) & = \mathcal{C}((\lambda\; \_ \to \mathtt{N})\; \mathtt{M}) \\
    \mathcal{C}(\mathtt{M \circ N}) &= \mathcal{C}(\mathtt{M});\; \mathcal{C}(\mathtt{N});\; \mathtt{bopToInstr}(\circ) \\
    \mathcal{C}(\mathtt{if\; B\; is\; 0\; then \; M\; else \; N}) &= \mathcal{C}(\mathtt{B});\; \mathtt{IF}\; (\mathcal{C}(\mathtt{M}); \mathtt{JOIN})\; (\mathcal{C}(\mathtt{N}); \mathtt{JOIN}) \\
    \mathcal{C}(\mathtt{fix\; (\lambda \ \_ \to \lambda \ \_ \to e)}) &= \mathtt{FIX}\; (\mathcal{C}(\mathtt{e}); \mathtt{RTN})
\end{aligned}
\]

Where \texttt{boptoInstr} is a function that transforms a binary operation $\circ$, of the form $ +, -, * $, into its associated instruction, and the underline denotes
the De Bruijn index of a variable. \\
We also don't care about the names the user has given to variables, as they get substituted by their De Bruijn indexes by the parser. 
The language syntax is shown here instead of the AST representation for readability.

\subsection{Interpreting Instructions}
Now that we have the abstract machine instructions, all that is left is to run them!

\begin{table}[h]
    \centering
    \small
    \setlength{\tabcolsep}{4pt}
    \begin{tabular}{|l|l|l|l||l|l|l|l|}
        \hline
        \multicolumn{4}{|c||}{\textbf{Before}} & \multicolumn{4}{c|}{\textbf{After}} \\
        \hline
        \textbf{Stack} & \textbf{Env} & \textbf{Code} & \textbf{Dump} & \textbf{Stack} & \textbf{Env} & \textbf{Code} & \textbf{Dump} \\
        \hline
        \( s \) & \( e \) & \( \text{LD} \ i : c \) & \( d \) & \( e[i] : s \) & \( e \) & \( c \) & \( d \) \\
        \( s \) & \( e \) & \( \text{LDC} \ k : c \) & \( d \) & \( \text{VInt} \ k : s \) & \( e \) & \( c \) & \( d \) \\
        \( \text{VInt} \ v_2 : \text{VInt} \ v_1 : s \) & \( e \) & \( \text{ADD} : c \) & \( d \) & \( \text{VInt} \ (v_1 + v_2) : s \) & \( e \) & \( c \) & \( d \) \\
        \( \text{VInt} \ v_2 : \text{VInt} \ v_1 : s \) & \( e \) & \( \text{SUB} : c \) & \( d \) & \( \text{VInt} \ (v_1 - v_2) : s \) & \( e \) & \( c \) & \( d \) \\
        \( \text{VInt} \ v_2 : \text{VInt} \ v_1 : s \) & \( e \) & \( \text{MUL} : c \) & \( d \) & \( \text{VInt} \ (v_1 * v_2) : s \) & \( e \) & \( c \) & \( d \) \\
        \hline
        \( s \) & \( e \) & \( \text{CLO} \ c' : c \) & \( d \) & \( \text{Addr} \ a : s \) & \( e \) & \( c \) & \( d \) \\
        \( s \) & \( e \) & \( \text{FIX} \ c' : c \) & \( d \) & \( \text{Addr} \ a : s \) & \( e \) & \( c \) & \( d \) \\
        \( v : \text{Addr} \ a : s \) & \( e \) & \( \text{AP} : c \) & \( d \) & \( [ \ ] \) & \( v : e' \) & \( c' \) & \( (s, e, c) : d \) \\
        \( v : s \) & \( e \) & \( \text{RTN} : c \) & \( (s', e', c') : d \) & \( v : s' \) & \( e' \) & \( c' \) & \( d \) \\
        \hline
        \( \text{VInt} \ 0 : s \) & \( e \) & \( \text{IF} \ c_1 \ c_2 : c \) & \( d \) & \( s \) & \( e \) & \( c_1 \) & \( ([ \ ], [ \ ], c) : d \) \\
        \( \text{VInt} \ (n \neq 0) : s \) & \( e \) & \( \text{IF} \ c_1 \ c_2 : c \) & \( d \) & \( s \) & \( e \) & \( c_2 \) & \( ([\ ], [ \ ], c) : d \) \\
        \( s \) & \( e \) & \( \text{JOIN} : c \) & \( (s', e', c') : d \) & \( s \) & \( e \) & \( c' \) & \( d \) \\
        \( b : s \) & \( e \) & \( \text{LET} : c \) & \( d \) & \( s \) & \( b : e \) & \( c \) & \( d \) \\
        \( s \) & \( e \) & \( \text{HALT} : c \) & \( d \) & \( s \) & \( e \) & \( [ \ ] \) & \( d \) \\
        \hline
    \end{tabular}
    \caption{SECD Machine Transitions}
    \label{table:secd}
\end{table}

The \textbf{Store} transitions are not shown in table \ref{table:secd}, but its easy to infer its transitions and use:
\begin{enumerate}
  \item When a function is created (using instructions like CLO or FIX), a closure is built and saved in 
    the store at a unique address (using the \texttt{Addr} value). This stored closure can later be 
    retrieved during function application;
  \item In the case of recursive functions, we store the function's own address inside the environment 
    so that it can refer to itself;
  \item Finally, for applications, one must retrieve the closure that is under the \texttt{Addr} that was popped
    from the stack. The closure's code is denoted as \( c' \) and the environment as \( e' \) in table \ref{table:secd}.
\end{enumerate}

