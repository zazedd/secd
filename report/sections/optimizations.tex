\section{Optimizations}
\subsection{Optimizing \texttt{let} declarations}
Recall that the original compilation for the \texttt{let} declarations was as follows:
\[ \mathcal{C}(\mathtt{let\; \_ = M\; in\; N}) = \mathcal{C}((\lambda\; \_ \to \mathtt{N})\; \mathtt{M}) \]

This creates an extra closure from the new application, which is not as efficient as it could be.
In fact, if we add a new instruction:
\begin{lstlisting}
data Instr
  = LD Int -- loads the variable at the specified index 
  ...
  | LET -- pops the top of the stack and prepends it to the environment
  ...
\end{lstlisting}

We can compile a \texttt{let} simply as:
\[ \mathcal{C}(\mathtt{let\; \_ = M\; in\; N}) = \mathcal{C}(M);\; \texttt{LET};\; \mathcal{C}(N) \]

And the interpreter can be extended as:

\begin{table}[h]
    \centering
    \small
    \setlength{\tabcolsep}{4pt}
    \begin{tabular}{|l|l|l|l||l|l|l|l|}
        \hline
        \multicolumn{4}{|c||}{\textbf{Before}} & \multicolumn{4}{c|}{\textbf{After}} \\
        \hline
        \textbf{Stack} & \textbf{Env} & \textbf{Code} & \textbf{Dump} & \textbf{Stack} & \textbf{Env} & \textbf{Code} & \textbf{Dump} \\
        \hline
        \( v : \ s \) & \( e \) & \( \text{LET} : c \) & \( d \) & \( s \) & \( v : e \) & \( c \) & \( d \) \\
        \hline
    \end{tabular}
    \caption{LET Transition}
\end{table}

It is easy to see that now the closest binded De Bruijn indexed variable, will be whatever \texttt{M} evaluates to. 
This models the behaviour we want for \texttt{let}.

\subsection{Modern SECD: Combining the Stack and the Dump}
You might have noticed that, when we push to the Dump, we always add the rest of the code left to run, and its environment. 
Doesn't that sound like a closure? Well, as it turns out, \textbf{it is}!
Let's implement the Dumpless, Storeless SECD machine, commonly called the CES machine.

\subsubsection{A Simpler Version First}

First things first, let's extend the \texttt{Val} data type to contain closures:
\begin{lstlisting}
data Val
  = VInt Int
  | VClos ([Instr], Env)
\end{lstlisting}

We also don't need \texttt{Addr} anymore, which was only used to access the \texttt{Store},
because we will now be keeping all closures on the stack.

Our compile function stays the same for now, the magic happens while interpreting! \cite{CockettAM}

\begin{table}[h]
    \centering
    \begin{tabular}{|l|l|l||l|l|l|}
        \hline
        \multicolumn{3}{|c||}{\textbf{Before}} & \multicolumn{3}{c|}{\textbf{After}} \\
        \hline
        \textbf{Code} & \textbf{Env} & \textbf{Stack} & \textbf{Code} & \textbf{Env} & \textbf{Stack} \\
        \hline
        LD \(n\) ; c & \( e \) & \( s \) & \( c \) & \( e \) & \( e(n) : s \) \\
        LDC \(k\) : c & \( e \) & \( s \) & \( c \) & \( e \) & \( k : s \) \\
        ADD : c & \( e \) & \( \text{VInt} \ v_2 : \text{VInt} \ v_1 : s \) & \( c \) & \( e \) & VInt \( (v_1 + v_2) : s \) \\
        SUB : c & \( e \) & \( \text{VInt} \ v_2 : \text{VInt} \ v_1 : s \) & \( c \) & \( e \) & VInt \( (v_1 - v_2) : s \) \\
        MUL : c & \( e \) & \( \text{VInt} \ v_2 : \text{VInt} \ v_1 : s \) & \( c \) & \( e \) & VInt \( (v_1 * v_2) : s \) \\
        \hline
        CLO \(c'\) : c & \( e \) & \( s \) & \( c \) & \( e \) & VClos (\( c' , e \)) : \( s \) \\
        AP : \(c\) & \( e \) & \( v \) : VClos (\( c' , e' \)) : \( s \) & \( c' \) & \( v : e' \) & VClos (\( c , e \)) : \( s \) \\
        RTN : c & \( e \) & \( v : \text{VClos} (c', e') : s \) & \( c' \) & \( e' \) & \( v : s \) \\
        \hline
        IF\ \( c_0 \ c_1 \) : c & \( e \) & \( \text{VInt} \ 0 : s \) & \( c_0 \) & \( e \) & Clos (\( c, e \)) : \( s \) \\
        IF\ \( c_0 \ c_1 \) : c & \( e \) & \( \text{VInt} \ (n \neq 0) : s \) & \( c_1 \) & \( e \) & Clos (\( c, e \)) : \( s \) \\
        \hline
        \( \text{LET} : c \) & \( e \) & \( v : \ s \) & \( c \) & \( v : e \) & \( s \) \\
        \( \text{HALT} : c \) & \( e \) & \( v : \ s \) & \( [ \ ] \) & \( e \) & \( s \) \\
        \hline
    \end{tabular}
    \caption{CES Machine Transitions, no FIX}
\end{table}

A side effect of this simplification is that we don't need a seperate instruction for joining the \texttt{if} branches
with the main execution. Because the \texttt{if} also creates a closure, we can return using \texttt{RTN} instead of \texttt{JOIN}.

You might have noticed that we are missing the fixed point operation. Lets tackle this hurdle.

\subsubsection{Adding the Fixed Point}

Instead of just the one case where our fixed point absolutely needed at least two abstractions inside (the recursive function itself, and a first argument), 
we can also handle the case where it only has one (just the recursive function). Let's call this case the \textit{control} case.

We can add a new constructor to our instruction set:
\begin{lstlisting}
data Instr
  ...
  | FIX [Instr] -- loads a recursive function
  | FIXC [Instr] -- loads a recursive function (control case)
  ...
\end{lstlisting}

And we should extend our \texttt{Val} data type to accommodate our new closures:

\begin{lstlisting}
data Val
  = VInt Int
  | VClos ([Instr], Env)
  | VFixClos ([Instr], Env) -- recursive closure
  | VFixCClos ([Instr], Env) -- recursive closure (control case)
\end{lstlisting}

We are now ready to both compile our new instructions, and interpret them!

The compilation cases are:
\[
\begin{aligned}
  \mathcal{C}(\mathtt{fix\; (\lambda \ \_ \to \lambda \ \_ \to e)}) & = \mathtt{FIX}\; (\mathcal{C}(\mathtt{e});\; \mathtt{RTN}) \\
  \mathcal{C}(\mathtt{fix\; (\lambda \ \_ \to e)}) & = \mathtt{FIXC}\; (\mathcal{C}(\mathtt{e});\; \mathtt{RTN})
\end{aligned}
\]
\begin{table}[h]
    \centering
    \small
    \begin{tabular}{|l|l|l||l|l|l|}
        \hline
        \multicolumn{3}{|c||}{\textbf{Before}} & \multicolumn{3}{c|}{\textbf{After}} \\
        \hline
        \textbf{Code} & \textbf{Env} & \textbf{Stack} & \textbf{Code} & \textbf{Env} & \textbf{Stack} \\
        \hline
        FIX \(c'\) : c & \( e \) & \( s \) & \( c \) & \( e \) & VFixClos (\( c' , e \)) : \( s \) \\
        AP : \(c\) & \( e \) & \( v \) : VFixClos (\( c' , e' \)) : \( s \) & \( c' \) & \( v \) : VFixClos (\( c' , e' \)) : \( e' \) & VClos (\( c , e \)) : \( s \) \\
        \hline
        FIXC \(c'\) : c & \( e \) & \( s \) & \( c \) & \( e \) & VFixCClos (\( c' , e \)) : \( s \) \\
        AP : \(c\) & \( e \) & \( v \) : VFixCClos (\( c' , e' \)) : \( s \) & \( c' \) & VFixClos (\( c' , e' \)) : \( e' \) & VClos (\( c , e \)) : \( s \) \\
        \hline
    \end{tabular}
    \caption{CES Machine Transitions for FIX}
    \label{table:ces}
\end{table}

Table \ref{table:ces} outlines the interpreting steps. \cite{CockettAM}

In short, the application for a fixed point not only applies the code, but also inserts the fixed point below
the argument of the application (if there is one) in the environment, so that a recursive call can be
executed correctly.

\subsection{Tail Call Optimization}
Now that we have successfuly eliminated the need for a Dump and a Store, doing tail call optimization is easy.

But what even are tail calls? Consider the following:

\[
\begin{aligned}
f &= \lambda x \to \dots \ g \ (x) \dots \\
g &= \lambda y \to h \ (...) \\
h &= \lambda z \to \dots
\end{aligned}
\]
The call from $g$ to $h$ is a \textbf{tail call}: when $h$ returns, $g$ has nothing more to
compute, it just returns immediately to $f$.

If we analyze the instructions this sequence would create, we would see that the code
for $g$ is of the form $ [\dots;\ \texttt{AP};\ \texttt{RTN}] $. This will create a closure with only
\texttt{RTN} inside, consuming stack space.

It may seem that this stack space is unimportant at first, but consider a recursive function
like the following one:

\begin{lstlisting}[mathescape]
let fact = fix ($ \lambda f \to \lambda n \to \lambda acc \to $ if n is 0 then acc else f (n - 1) (acc * n))
in fact 42 1
\end{lstlisting}

The recursive call to $ f $ is in tail position. If we don't eliminate the tail call,
this code will run with $\mathcal{O}(n)$ stack space, which risks a stack overflow. If we
do apply tail call optimization, it runs in $\mathcal{O}(1)$ stack space as expected.

So what is the trick? First, we need a new instruction specifically for tail applications.
Then, we should split the compilation into two mutually recursive functions, $\mathcal{C}$ and $\mathcal{T}$, for normal terms
and terms in tail call position respectively.

\newpage

We also need a new instruction to signal the end of a \texttt{let} declaration:

\begin{lstlisting}
data Instr
  = LD Int -- loads the variable at the specified index 
  ...
  | TAP -- tail application
  | ENDLET
  ...
\end{lstlisting}

Now $\mathcal{C}$ and $\mathcal{T}$ \cite{LeroyMPRI}:
\[
\begin{aligned}
    \mathcal{C}(n) &= \mathtt{LDC}\; n \\
    \mathcal{C}(\underline{n}) &= \mathtt{LD}\; n \\
    \mathcal{C}(\mathtt{\lambda\; \_ \to M}) &= \mathtt{CLO}\; (\textcolor{red}{\mathcal{T}(\mathtt{M})}) \\
    \mathcal{C}(\mathtt{M\; N}) &= \mathcal{C}(\mathtt{M});\; \mathcal{C}(\mathtt{N});\; \mathtt{AP} \\
    \mathcal{C}(\mathtt{let\; \_ = M\; in\; N}) & = \mathcal{C}(\mathtt{M});\; \texttt{LET};\; \mathcal{C}(\mathtt{N});\; \texttt{ENDLET} \\
    \mathcal{C}(\mathtt{M \circ N}) &= \mathcal{C}(\mathtt{M});\; \mathcal{C}(\mathtt{N});\; \mathtt{bopToInstr}(\circ) \\
    \mathcal{C}(\mathtt{if\; B\; is\; 0\; then \; M\; else \; N}) &= \mathcal{C}(\mathtt{B});\; \mathtt{IF}\; \left(\textcolor{red}{\mathcal{T}(\mathtt{M})}\right)\; \left(\textcolor{red}{\mathcal{T}(\mathtt{N})}\right) \\
    \mathcal{C}(\mathtt{fix\; (\lambda \ \_ \to \lambda \ \_ \to e)}) &= \mathtt{FIX}\; \left(\textcolor{red}{\mathcal{T}(\mathtt{e})}\right) \\
    \mathcal{C}(\mathtt{fix\; (\lambda \ \_ \to e)}) &= \mathtt{FIXC}\; \left(\textcolor{red}{\mathcal{T}(\mathtt{e})}\right) \\
    \\
    \mathcal{T}(\mathtt{let\; \_ = M\; in\; N}) & = \mathcal{C}(\mathtt{M});\; \texttt{LET};\; \mathcal{T}(\mathtt{N}) \\
    \mathcal{T}(\mathtt{M\; N}) &= \mathcal{C}(\mathtt{M});\; \mathcal{C}(\mathtt{N});\; \mathtt{TAP} \\
    \mathcal{T}(a) &= \mathcal{C}(a);\; \texttt{RTN}
\end{aligned}
\]
Now for the semantics of \texttt{TAP}, it is very simple. They are a mirror of the \texttt{AP} rules, except that they don't bother
to push a new closure onto the stack.

\begin{table}[h]
    \centering
    \small
    \begin{tabular}{|l|l|l||l|l|l|}
        \hline
        \multicolumn{3}{|c||}{\textbf{Before}} & \multicolumn{3}{c|}{\textbf{After}} \\
        \hline
        \textbf{Code} & \textbf{Env} & \textbf{Stack} & \textbf{Code} & \textbf{Env} & \textbf{Stack} \\
        \hline
        TAP : \(c\) & \( e \) & \( v \) : Clos (\( c' , e' \)) : \( s \) & \( c' \) & \( v \) : \( e' \) & \( s \) \\
        TAP : \(c\) & \( e \) & \( v \) : VFixClos (\( c' , e' \)) : \( s \) & \( c' \) & \( v \) : VFixClos (\( c' , e' \)) : \( e' \) & \( s \) \\
        TAP : \(c\) & \( e \) & \( v \) : VFixCClos (\( c' , e' \)) : \( s \) & \( c' \) & VFixClos (\( c' , e' \)) : \( e' \) & \( s \) \\
        \hline
    \end{tabular}
    \caption{TAP Machine Transitions}
\end{table}

\newpage

Table \ref{table:stacks} explores the differences between our tail optimized compiler, and our last compiler definition. Specifically, it looks
at the maximum number of items on the stack for an evaluation of the \texttt{fact 42 1} term above.

You can see how this scales exponentially.

\begin{table}[h]
    \centering
    \small
    \begin{tabular}{|r||c|c|}
        \hline
        Step \# & Non-TO & TO \\
        \hline \hline
        ... & ... & ... \\
        \hline
        552 & 86 & 45 \\
        \hline
        553 & 87 & 46 \\
        \hline
        554 & 86 & 45 \\
        \hline
        555 & 85 & 43 \\
        \hline
        556 & 86 & 44 \\
        \hline
        557 & 86 & 44 \\
        \hline
        558 & 87 & 45 \\
        \hline
        ... & ... & ... \\
        \hline
        603 & 42 & 1 \\
        \hline
        ... & ... & 0 \\
        \hline
        645 & 1 & 0 \\
        \hline
    \end{tabular}
    \caption{Stack Sizes: Non-Tail optimized (Non-TO) vs. Tail optimized (TO)}
    \label{table:stacks}
\end{table}
